% ************************** Thesis Abstract *****************************
% Use `abstract' as an option in the document class to print only the titlepage and the abstract.
\begin{abstract}
Artificial Neural Networks are connectionist systems that perform a given task by learning on examples without having prior knowledge about the task. This is done by finding an optimal point estimate for the weights in every node.
Generally, the network using point estimates as weights perform well with large datasets, but they fail to express uncertainty in regions with little or no data, leading to overconfident decisions.
\newline

In this thesis, Bayesian Convolutional Neural Network (BayesCNN) using Variational Inference is proposed, that introduces probability distribution over the weights. Furthermore, the proposed BayesCNN architecture is applied to tasks like Image Classification, Image Super-Resolution and Generative Adversarial Networks.

BayesCNN is based on Bayes by Backprop which derives a variational approximation to the true posterior. 
Our proposed method not only achieves performances equivalent to frequentist inference in identical architectures but also incorporate a measurement for uncertainties and regularisation. It further eliminates the use of dropout in the model. Moreover, we predict how certain the model prediction is based on the epistemic and aleatoric uncertainties and finally, we propose ways to prune the Bayesian architecture and to make it more computational and time effective. 
\newline


In the first part of the thesis, the Bayesian Neural Network is explained and it is applied to an Image Classification task. The results are compared to point-estimates based architectures on MNIST, CIFAR-10, and CIFAR-100 datasets. Moreover, uncertainties are calculated and the architecture is pruned and a comparison between the results is drawn.

In the second part of the thesis, the concept is further applied to other computer vision tasks namely, Image Super-Resolution and Generative Adversarial Networks. The concept of BayesCNN is tested and compared against other concepts in a similar domain.  

\end{abstract}
