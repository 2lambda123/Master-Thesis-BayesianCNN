%!TEX root = ../thesis.tex
%*******************************************************************************
%****************************** Third Chapter **********************************
%*******************************************************************************
\chapter{Related Work}

% **************************** Define Graphics Path **************************
\ifpdf
    \graphicspath{{Chapter3/Figs/Raster/}{Chapter3/Figs/PDF/}{Chapter3/Figs/}}
\else
    \graphicspath{{Chapter3/Figs/Vector/}{Chapter3/Figs/}}
\fi

\fbox{
    \parbox{\textwidth}
    {
        Chapter Overview
        \begin{itemize}
            \item How Bayesian Methods were applied to Neural Networks for the intractable true posterior distribution.
            \item Various ways of training Neural Networks posterior probability distributions: Laplace approximations, Monte Carlo and Variational Inference.
            \item Proposals on Dropout and Gaussian Dropout as Variational Inference schemes.
            \item Work done in the past for uncertainty estimation in Neural Network.
            \item Ways to reduce the number of parameters in a model.
        \end{itemize}
    }
}

\pagebreak

\section{Related Work}

\subsection{Bayesian Training}
Applying Bayesian methods to neural networks has been studied in the past with various approximation methods for the intractable true posterior probability distribution $p(w|\mathcal{D})$. Buntine and Weigend \cite{buntine1991bayesian} started to propose various \textit{maximum-a-posteriori} (MAP) schemes for neural networks. They were also the first who suggested second order derivatives in the prior probability distribution $p(w)$ to encourage smoothness of the resulting approximate posterior probability distribution.
In subsequent work by Hinton and Van Camp \cite{hinton1993keeping}, the first variational methods were proposed which naturally served as a regularizer in neural networks. He also mentioned that the amount of information in a weight can be controlled by adding Gaussian noise. When optimizing the trade-off between the expected error and the information in the weights, the noise level can be adapted during learning.


Hochreiter and Schmidhuber \cite{hochreiter1995simplifying} suggested taking an information theory perspective into account and utilising a minimum description length (MDL) loss. This penalises non-robust weights by means of an approximate penalty based upon perturbations of the weights on the outputs.
Denker and LeCun \cite{denker1991transforming}, and MacKay \cite{mackay1995probable} investigated the posterior probability distributions of neural networks and treated the search in the model space (the space of architectures, weight decay, regularizers, etc..) as an inference problem and tried to solve it using Laplace approximations.
As a response to the limitations of Laplace approximations, Neal \cite{neal2012bayesian} investigated the use of hybrid Monte Carlo for training neural networks, although it has so far been difficult to apply these to the large sizes of neural networks built in modern applications. 
Also, these approaches lacked scalability in terms of both the network architecture and the data size. 


More recently, Graves \cite{graves2011practical} derived a variational inference scheme for neural networks and Blundell et al. \cite{blundell2015weight} extended this with an update for the variance that is unbiased and simpler to compute. Graves \cite{graves2016stochastic} derives a similar algorithm in the case of a mixture posterior probability distribution. 
A more scalable solution based on expectation propagation was proposed by Soudry \cite{Soudry:NIPS2014_5269} in 2014. While this method works for networks with binary weights, its extension to continuous weights is unsatisfying as it does not produce estimates of posterior variance.

\newline Several authors have claimed that Dropout \cite{srivastava2014dropout} and Gaussian Dropout \cite{wang2013fast} can be viewed as approximate variational inference schemes \cite{gal2015bayesian, kingma2015variational}. We compare our results to Gal's \& Ghahramani's \cite{gal2015bayesian} and discuss the methodological differences in detail.


\subsection{Uncertainty Estimation}

Neural Networks can predict uncertainty when Bayesian methods are introduced in it. An attempt to model uncertainty has been studied from 1990s \cite{neal2012bayesian} but has not been applied successfully until 2015. Gal and Ghahramani \cite{Gal2015Dropout} in 2015 provided a theoretical framework for modelling Bayesian uncertainty. Gal and Ghahramani \cite{gal2015bayesian} obtained the uncertainty estimates by casting dropout training in conventional deep networks as a Bayesian approximation of a Gaussian Process. They showed that any network trained with dropout is an approximate Bayesian model, and uncertainty estimates can be obtained by computing the variance on multiple predictions with different dropout masks.

\subsection{Model Pruning}

Some early work in the model pruning domain used a second-order Taylor approximation of the increase in the loss function of the network when a weight is set to zero \cite{lecun1990optimal}. A diagonal Hessian approximation was used to calculate the saliency for each parameter \cite{lecun1990optimal} and the low-saliency parameters were pruned from the network and the network was retrained.


Narang \cite{DBLP:journals/corr/NarangDSE17} showed that a pruned RNN and GRU model performed better for the task of speech recognition compared to a dense network of the original size. This result is very similar to the results obtained in our case where a pruned model achieved better results than a normal network. However, no comparisons can be drawn as the model architecture (CNN vs RNN) used and the tasks (Computer Vision vs Speech Recognition) are completely different.  Narang \cite{DBLP:journals/corr/NarangDSE17} in his work introduced a gradual pruning scheme based on pruning all the weights in a layer
less than some threshold (manually chosen) which is linear with some
slope in phase 1 and linear with some slope in phase 2 followed by
normal training. However, we reduced the number of filters to half for one case and in the other case, we induced a sparsity-based on L1 regularization to remove the less contributing weights and reduced the parameters. 


Other similar work \cite{DBLP:journals/corr/AnwarHS15, DBLP:journals/corr/LebedevGROL14, DBLP:journals/corr/ChangpinyoSZ17} to our work that reduces or removed the redundant connections or induces sparsity are motivated by the desire to speed up computation.
The techniques used are highly convolutional layer dependent and is not applicable to other architectures like Recurrent Neural Networks. 
One another interesting method of pruning is to represent each parameter with a smaller floating point number like 16-bits instead of 64 bits. This way there is a speed up in the training and inference time and the model is less computationally expensive. 

Another viewpoint for model compression was presented by Gong \cite{DBLP:journals/corr/GongLYB14}. He proposed vector quantization to achieve different compression ratios and different accuracies and depending on the use case, the compression and accuracies can be chosen. However, it requires a different hardware architecture to support the inference at runtime. Besides quantization, other potentially complementary approaches to reducing model size include low-rank matrix factorization \cite{denton2014exploiting, DBLP:journals/corr/LebedevGROL14} and group sparsity regularization to arrive at an optimal layer size \cite{DBLP:conf/nips/AlvarezS16}.

